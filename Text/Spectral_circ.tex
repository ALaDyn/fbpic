\documentclass[a4paper]{article}   	% use "amsart" instead of "article" for AMSLaTeX format

\usepackage{geometry}               	
\geometry{a4paper,margin=1.1in}           % ... or a4paper or a5paper or ... 

%\usepackage{graphicx}			

\usepackage{amsmath}

\usepackage{hyperref}
\hypersetup{colorlinks=true,
    citecolor=blue,    filecolor=blue,
    linkcolor=blue,     urlcolor=blue
}
\usepackage{cite}

\newcommand{\ir}{\frac{1}{r}}
\newcommand{\Integ}[1]{\int_{-\infty}^{\infty} \!\!\!\!\!
  \mathrm{d}#1}
\newcommand{\RInteg}[1]{\int_{0}^{\infty} \!\!\!\!\! #1\mathrm{d}#1}
\newcommand{\TInteg}[1]{\int_{0}^{2\pi} \!\!\!\!\! \mathrm{d}#1}


% Reference automatique aux equations, sections, figures
\usepackage{cleveref}

\title{A pseudo-spectral, cylindrical and dispersion-free Particle-In-Cell algorithm}
\date{}
%\pagestyle{empty}
\author{R\'emi \textsc{Lehe}}

% Mettre les c2 partout

\begin{document}

\maketitle


\section*{Introduction}

\paragraph{Importance of PIC codes } Used in many fields : astro, laser-plasma interaction. However, 3D codes are expensive, and their limited resolution leads to many numerical effects (Cherenkov, dispersion).

\paragraph{Cylindrical PIC codes} A range of physical situations have close-to cylindrical symmetry (e.g. electron beam propagation, laser-wakefield acceleration), and they can be simulated with very significant speedup using a cylindrically symmetric algorithm (Lischitz, Davidson). This typically reduces the cost of the simulation to a few times that of a 2D simulation, instead of that of a full 3D simulation.

\paragraph{Spectral PIC codes} In parallel, 3D spectral PIC codes have been developped, and can mitigate numerical instability (articles Warren and Brendan Godfrey). In spectral space, some algorithms have no numerical dispersion whatsoever (article PSATD). This is important in laser-wakefield acceleration since even a very small amount of numerical dispersion can introduce spurious injection. Typically, this imposes to strongly overresolve the wavelength of the laser (thereby increasing computational requirements), and to use very asymetric cell aspect ratio. Although the spectral algorithms reduce the requirements on the longitudinal resolution, they still require a 3D grid, and therefore a large computational load.

\paragraph{Spectral, cylindrical PIC codes } In this document, we combine the two approaches. Advantages:
\begin{itemize}
\item Significant speedup due to the cylindrical geometry.
\item Ideal dispersion relation in PSATD, independent on azimuthal mode (with a finite difference algorithm, the different modes go at different speeds)
\item Fields are colocated in space and time in PSATD and not
  staggered, which simplifies interpolation, and prevents
  time-interpolation errors in compensation of E and B (article plasma
  lens)
\item Time step can be bigger (but still need to resolve the
  oscillations of the particles in time).
\item Ability to apply filters in k space and thereby limit numerical instabilities
\item Prevent on-axis noise that appear in finite-difference cylindrical algorithms.
\end{itemize}


\section{Representation of the fields and continuous equations}

\subsection{Reminder on Cartesian spectral codes}

It is well-known that the Maxwell equations in Cartesian coordinates 
\begin{align}
\partial_t E_x = \partial_y B_z - \partial_z B_y - \mu_0  j_x \qquad&   
\partial_t B_x = -\partial_y E_z + \partial_z E_y \label{eq:CartMaxwellx} \\
\partial_t E_y = \partial_z B_x - \partial_x B_z - \mu_0  j_y \qquad &   
\partial_t B_y = -\partial_z E_x + \partial_x E_z \label{eq:CartMaxwelly}  \\
\partial_t E_z = \partial_x B_y - \partial_y B_x - \mu_0  j_z \qquad &   
\partial_t B_z = -\partial_x E_y + \partial_y E_x \label{eq:CartMaxwellz} 
\end{align}
can be solved by representing the fields as a sum of Fourier modes.
\begin{equation}
\label{eq:CartBwTrans}
F_u(\vec{r}) = \frac{1}{(2\pi)^{3/2}}\Integ{k_x} \,\Integ{k_y}\, \Integ{k_z} \; \hat{F}_u(\vec{k}) e^{i(k_x x + k_y y + k_z z)} 
\end{equation}
with 
\begin{equation}
\label{eq:CartFwTrans}
\hat{F}_u(\vec{k})  = \frac{1}{(2\pi)^{3/2}}\Integ{x} \,\Integ{y}\, \Integ{z} \; F_u(\vec{r}) e^{-i(k_x x + k_y y + k_z z)} 
\end{equation}
where $F$ is any of the fields $E$, $B$ or $j$, and where $u$ is
either $x$, $y$ or $z$. In this case, the different Fourier modes decouple and the equations \cref{eq:CartMaxwellx,eq:CartMaxwelly,eq:CartMaxwellz} become 
\begin{align}
\partial_t \hat{E}_x = ik_y \hat{B}_z - ik_z \hat{B}_y - \mu_0 \hat{j}_x \qquad &   
\partial_t \hat{B}_x = -ik_y \hat{E}_z + ik_z \hat{E}_y \\
\partial_t \hat{E}_y = ik_z \hat{B}_x - ik_x \hat{B}_z - \mu_0  \hat{j}_y \qquad &   
\partial_t \hat{B}_y = -ik_z \hat{E}_x + ik_x \hat{E}_z \\
\partial_t \hat{E}_z = ik_x \hat{B}_y - ik_y \hat{B}_x - \mu_0 \hat{j}_z  \qquad &   
\partial_t \hat{B}_z = -ik_x \hat{E}_y + ik_y \hat{E}_x 
\end{align}
The Fourier coefficients can then be integrated in time, and
transformed back into real space using \cref{eq:CartBwTrans}. This is
the core principle of Cartesian pseudo-spectral codes.

\subsection{Cylindrical spectral decomposition}

The Fourier representation \cref{eq:CartBwTrans} is no longer the
appropriate representation when the Maxwell equations are written in cylindrical coordinates.
\begin{align}
\partial_t E_r = \ir \partial_\theta B_z - \partial_z B_\theta - \mu_0  j_r \qquad&   
\partial_t B_r = -\ir \partial_\theta E_z + \partial_z E_\theta \label{eq:CircMaxwellr} \\
\partial_t E_\theta = \partial_z B_r - \partial_r B_z - \mu_0  j_\theta \qquad &   
\partial_t B_\theta = -\partial_z E_r + \partial_r E_z \label{eq:CircMaxwellt}  \\
\partial_t E_z = \ir\partial_r r B_\theta - \ir\partial_\theta B_r - \mu_0  j_z \qquad &   
\partial_t B_z = -\ir\partial_r r E_\theta + \ir\partial_\theta E_r \label{eq:CircMaxwellzz} 
\end{align}
When replacing
\cref{eq:CartBwTrans} into the \cref{eq:CircMaxwellr,eq:CircMaxwellt,eq:CircMaxwellzz}, the Fourier modes do not decouple. Instead one has to use the Fourier-Bessel representation.
\begin{align}
& F_u(\vec{r}) = \sum_{m=-\infty}^{\infty} \Integ{k_z}
\RInteg{k}\; \tilde{F}_{u,m}(k_z,k) \; J_m(kr)\, e^{-im\theta + ik_z z} 
\label{eq:CircBwTransu} \\
& F_r(\vec{r}) = \sum_{m=-\infty}^{\infty} \Integ{k_z}\,\RInteg{k}\;
\left( \tilde{F}_{+,m}(k_z,k)\; J_{m+1}(kr) +\tilde{F}_{-,m}(k_z,k)\; J_{m-1}(kr)
\right)  e^{-im\theta +ik_z z}
\label{eq:CircBwTransr} \\
& F_\theta(\vec{r}) = \sum_{m=-\infty}^{\infty} \Integ{k_z}\,\RInteg{k}\;
i\left( \tilde{F}_{+,m}(k_z,k)\; J_{m+1}(kr) - \tilde{F}_{-,m}(k_z,k)\; J_{m-1}(kr)
\right)  e^{-im\theta +ik_z z} 
\label{eq:CircBwTranst}
\end{align}
where $F$ is either $E$, $B$ or $j$ and $u$ is any of the
Cartesian components $x$, $y$ or $z$. In addition
\begin{align}
\tilde{F}_{u,m}(k_z,k) = \frac{1}{(2\pi)^2} \Integ{z} \RInteg{r}
\TInteg{\theta} \;F_u(\vec{r})\; J_m(kr) e^{-im\theta
 - i k_z z} \label{eq:CircFwTransu} \\
\tilde{F}_{+,m} = \frac{\tilde{F}_{x,m+1} -
    i\tilde{F}_{y,m+1}}{2} \qquad \tilde{F}_{-,m} = \frac{\tilde{F}_{x,m-1} +
    i\tilde{F}_{y,m-1}}{2}
\label{eq:CircFwTranspm} 
\end{align}
\noindent (See \cref{sec:CircTrans} for a derivation of this representation and
 for its relation to the Fourier transform.)

When replacing \cref{eq:CircBwTransr,eq:CircBwTranst,eq:CircBwTransu}
(for $u=z$) into the Maxwell equations in cylindrical
coordinates \cref{eq:CircMaxwellr,eq:CircMaxwellt,eq:CircMaxwellzz},
the different modes decouple, and the equations for the spectral
coefficients become:
\begin{align}
\partial_t \tilde{E}_{+,m} = - \frac{ik}{2}\tilde{B}_{z,m} + k_z\tilde{B}_{+,m} - \mu_0\tilde{j}_{+,m} \qquad &   
\partial_t \tilde{B}_{+,m} = \frac{ik}{2} \tilde{E}_{z,m} - k_z
\tilde{E}_{+,m} 
\label{eq:CircMaxwellp} \\
\partial_t \tilde{E}_{-,m} = -\frac{ik}{2} \tilde{B}_{z,m} - k_z \tilde{B}_{-,m} - \mu_0  \tilde{j}_{-,m} \qquad &   
\partial_t \tilde{B}_{-,m} = \frac{ik}{2} \tilde{E}_{z,m} + k_z
\tilde{E}_{-,m} \label{eq:CircMaxwellm} \\
\partial_t \tilde{E}_{z,m} = ik \tilde{B}_{+,m} + ik\tilde{B}_{-,m}  - \mu_0 \tilde{j}_{z,m}  \qquad & 
\partial_t \tilde{B}_{z,m} = -ik \tilde{E}_{+,m} - ik\tilde{E}_{-,m}  \label{eq:CircMaxwellz} 
\end{align}
(See \cref{sec:SpectMaxwell} for a derivation of these equations.) In addition, the conservation equations
\begin{equation}
\vec{\nabla}\cdot\vec{E} \equiv \ir \partial_r rE_r + \ir\partial_\theta E_\theta + \partial_z E_z
= \frac{\rho}{\epsilon_0} \qquad  \vec{\nabla}\cdot\vec{B} \equiv \ir \partial_r rB_r + \ir\partial_\theta B_\theta + \partial_z B_z
= 0 
\end{equation}
become
\begin{equation}
k(\tilde{E}_{+,m} -\tilde{E}_{-,m}) + ik_z \tilde{E}_{z,m} =
\frac{\tilde{\rho}}{\epsilon_0} \qquad
 k(\tilde{B}_{+,m} -\tilde{B}_{-,m}) + ik_z \tilde{B}_{z,m} =
0 \end{equation}

Again, the equations \cref{eq:CircMaxwellr,eq:CircMaxwellt,eq:CircMaxwellz}  can be
integrated in time, and the fields can then be transformed back into
real space, using \cref{eq:CircBwTransu,eq:CircBwTransr,eq:CircBwTranst}. Typically one needs only a few modes in $m$, therefore the
fields are reduced to a few 2D tables $\tilde{F}_m(k_z,k)$ instead of the 3D tables
$\hat{F}(k_x,k_y,k_z)$. (For instance, in a purely symmetric case, there is only one
mode: $m=0$.)

+ Mention the complex conjugates : only $m \geq 0$.


\section{Numerical implementation}

Grand scheme :
Explain intermediate grid (x,y,z)
3 grids, interpolation, deposition 
Explain shape of the particle

\subsection{Field integration}

Could be integrated by finite difference, but we choose PSATD (for numerical dispersion).

\subsection{Interpolation} 

\subsection{Current deposition}


\newpage
\appendix


\section{Derivation of the Fourier-Bessel representation}
\label{sec:CircTrans}

In order to derive the representation \cref{eq:CircBwTrans} we have to
distinguish the fields that are well-defined everywhere in space (like
$E_x$, $E_y$, $E_z$, $B_x$, $B_y$, $B_z$) and thus have a regular
Fourier representation, from those that are ill-defined at $r=0$ (like $E_r$, $E_\theta$, $B_r$, $B_\theta$).

\subsection{Fields that are well-defined everywhere}

Let $F_u$ be a field that is well-defined everywhere in space
(typically $F$ is $E$, $B$ or $J$ and $u$ is $x$, $y$ or $z$). Its Fourier representation
is thus given by \cref{eq:CartBwTrans,eq:CartFwTrans}:
\begin{align*}
F_u(\vec{r}) = \frac{1}{(2\pi)^{3/2}}\Integ{k_x} \,\Integ{k_y}\,
\Integ{k_z} \; \hat{F_u}(\vec{k}) e^{i(k_x x + k_y y + k_z z)} \\
\hat{F_u}(\vec{k})  = \frac{1}{(2\pi)^{3/2}}\Integ{x} \,\Integ{y}\,
\Integ{z} \; F_u(\vec{r}) e^{-i(k_x x + k_y y + k_z z)} 
\end{align*}
Using the change of variable $k_x=k_\perp\cos(\phi)$, $k_y = k_\perp\sin(\phi)$,
$x=r\cos(\theta)$, $y=r\sin(\theta)$, this becomes
 \begin{align*}
F_u(\vec{r}) = \frac{1}{(2\pi)^{3/2}}\Integ{k_z} \,\RInteg{k_\perp}\,
\TInteg{\phi} \; \hat{F_u}(\vec{k})
e^{i(k_\perp r \cos(\theta-\phi) + k_z z)} \\
\hat{F_u}(\vec{k})   = \frac{1}{(2\pi)^{3/2}}\Integ{z} \,\RInteg{r}\,
\Integ{\theta} \; F_u(\vec{r}) e^{-i(k_\perp r \cos(\theta-\phi) + k_z z)} 
\end{align*}
We now use the relation $e^{ik_\perp r\cos(\theta-\phi)} =
\sum_{m=-\infty}^{\infty} i^m J_m(k_\perp r) e^{im(\phi-\theta)}$ in these equations, to obtain
\begin{align*}
F_u (\vec{r})  = \sum_{m=-\infty}^{\infty} \frac{1}{(2\pi)^{3/2}}\Integ{k_z} \,\RInteg{k_\perp }
\TInteg{\phi} \; i^m \hat{F_u}(\vec{k}) \;
J_m(k_\perp r) e^{-im(\theta-\phi) + ik_z z} \\
\hat{F_u}(\vec{k})   =  \sum_{m=-\infty}^{\infty} \frac{1}{(2\pi)^{3/2}}\Integ{z} \,\RInteg{r}
\TInteg{\theta} \;\; (-i)^m F_u(\vec{r}) \; J_m(k_\perp r) e^{-im(\phi-\theta) -ik_z z} 
\end{align*}
We now define $\tilde{F}_{u,m}(k_z,k_\perp ) = \frac{1}{(2\pi)^{3/2}}\int_0^{2\pi}
\mathrm{d}\phi \; i^m \hat{F_u}(\vec{k})
e^{im\phi}$. This results in the following equations :
\begin{align*}
F_u(\vec{r}) = \sum_{m=-\infty}^{\infty} \Integ{k_z}
\RInteg{k_\perp }\; \tilde{F}_{u,m}(k_z,k_\perp ) \; J_m(k_\perp r) e^{-im\theta + ik_z z} 
\\
\tilde{F}_{u,m}(k_z,k_\perp ) = \frac{1}{(2\pi)^2} \Integ{z} \RInteg{r}
\TInteg{\theta} \;F_u(\vec{r})\; J_m(k_\perp r) e^{-im\theta
 - i k_z z}
\end{align*}
These equations correspond to \cref{eq:CircBwTransz,eq:CircFwTransz}.

\subsection{Fields that are ill-defined at $r=0$}

Let us now consider fields of the type $E_r$, $B_r$ or $J_r$, which we
denote generally by $F_r$. We have :
\[ F_r = \cos(\theta) F_x + \sin(\theta) F_y 
= \frac{F_x - iF_y}{2}e^{i\theta} + \frac{F_x +
  iF_y}{2}e^{-i\theta} \] 
Using \cref{eq:CircBwTransu,eq:CircFwTransu}, this leads to
\begin{align} 
F_r =  \sum_{m=-\infty}^{\infty} \Integ{k_z}\,\RInteg{k_\perp }\;
\left(  J_m(k_\perp r) \frac{\tilde{F}_{x,m} -
    i\tilde{F}_{y,m}}{2}e^{-i(m-1)\theta +ik_z z} + J_m(k_\perp r)
  \frac{\tilde{F}_{x,m} +   i\tilde{F}_{y,m}}{2}e^{-i(m+1)\theta +
    ik_z z} \right) \\
= \sum_{m=-\infty}^{\infty} \Integ{k_z}\,\RInteg{k_\perp }\;
\left(  J_{m+1}(k_\perp r) \frac{\tilde{F}_{x,m+1} -
    i\tilde{F}_{y,m+1}}{2}e^{-im\theta +ik_z z} + J_{m-1}(k_\perp r)
  \frac{\tilde{F}_{x,m-1} +   i\tilde{F}_{y,m-1}}{2}e^{-im\theta +
    ik_z z} \right) 
\end{align}
where we relabeled the dummy variable $m$ in the above sums. Let us
thus define $\tilde{F}_{-,m} = \frac{\tilde{F}_{x,m-1} +
    i\tilde{F}_{y,m-1}}{2}$ and $\tilde{F}_{+,m} = \frac{\tilde{F}_{x,m+1} -
    i\tilde{F}_{y,m+1}}{2}$. This results in:
\begin{equation} 
F_r(\vec{r}) = \sum_{m=-\infty}^{\infty} \Integ{k_z}\,\RInteg{k_\perp }\;
\left( \tilde{F}_{+,m}\; J_{m+1}(k_\perp r) +\tilde{F}_{-,m}\; J_{m-1}(k_\perp r)
\right)  e^{-im\theta +ik_z z}
\end{equation}

With the same definitions and the same method, it is also easy to show that:
\begin{equation} 
F_\theta(\vec{r}) = \sum_{m=-\infty}^{\infty} \Integ{k_z}\,\RInteg{k_\perp }\;
i\left( \tilde{F}_{+,m}\; J_{m+1}(k_\perp r) - \tilde{F}_{-,m}\; J_{m-1}(k_\perp r)
\right)  e^{-im\theta +ik_z z}
\end{equation}

\section{Maxwell equations for the spectral coefficients}
\label{sec:SpectMaxwell}

In this section, let us derive the Maxwell equations for the spectral
coefficients \cref{eq:CircMaxwellp,eq:CircMaxwellm,eq:CircMaxwellz}
from the Maxwell equations written in cylindrical coordinates \cref{eq:CircMaxwellr,eq:CircMaxwellt,eq:CircMaxwellzz}.

When replacing the Fourier-Bessel decomposition
(\cref{eq:CircBwTransu,eq:CircBwTransr,eq:CircBwTranst}) in the
Maxwell equations \cref{eq:CircMaxwellr,eq:CircMaxwellt,eq:CircMaxwellzz}, we
first notice that the modes proportional to $e^{-im\theta +ik_z z}$ for different
values of $m$ and $k_z$ are not coupled. These different modes can
thus be treated separately. The same cannot be said of the modes
corresponding to different values of $k_\perp $, since they may be coupled
through the Bessel functions $J_m(k_\perp r)$ and their derivatives. 
In the following, we write only the equations corresponding to $\partial_t \vec{B} =
-\vec{\nabla}\times \vec{E}$, since the equation $c^{-2}\partial_t \vec{E} =
\vec{\nabla}\times\vec{B} - \mu_0 \vec{j}$ can be treated very
similarly. These equations become
\begin{align*}
\RInteg{k_\perp } \left[ \; \partial_t \tilde{B}_{+,m}  J_{m+1}(k_\perp r)
  + \partial_t \tilde{B}_{-,m}  J_{m-1}(k_\perp r) \; \right] =&& \\ 
\qquad \RInteg{k_\perp } \left[ \; \tilde{E}_{z,m} \frac{im}{r}
  J_m(k_\perp r) \right.-&\left.
  k_z\tilde{E}_{+,m}J_{m+1}(k_\perp r) + k_z\tilde{E}_{-,m}J_{m-1}(k_\perp r) \;
\right] & \\
\RInteg{k_\perp } \left[ \; \partial_t \tilde{B}_{+,m}  J_{m+1}(k_\perp r)
  - \partial_t \tilde{B}_{-,m}  J_{m-1}(k_\perp r) \; \right] =&& \\
 \RInteg{k_\perp } \left[ \; -k_z\tilde{E}_{+,m}J_{m+1}(k_\perp r)
 \right.-&\left.  k_z\tilde{E}_{-,m}J_{m-1}(k_\perp r) - ik_\perp \tilde{E}_{z,m} J_m'(k_\perp r) \;\right] \\
\RInteg{k_\perp }\; \partial_t \tilde{B}_{z,m}  J_{m}(k_\perp r) =
\RInteg{k_\perp } \left[ \; -ik_\perp
  \tilde{E}_{+,m}\right.&\left(\frac{J_{m+1}(k_\perp r)}{k_\perp r} +
    J_{m+1}'(k_\perp r) \right) + &\\
\left. ik_\perp \tilde{E}_{-,m}\left(\frac{J_{m-1}(k_\perp r)}{k_\perp r} +
    J_{m-1}'(k_\perp r) \right) \right.-&\left. \frac{im}{r} \left( E_{+,m} J_{m+1}(k_\perp r) +
    E_{-,m} J_{m-1}(k_\perp r) \right) \;\right] 
\end{align*}
By taking the sum and difference of the first two equations, and by
rearranging the third equation, we obtain:
\begin{align*}
\RInteg{k_\perp } \; 2 \,\partial_t \tilde{B}_{+,m}  J_{m+1}(k_\perp r) =
\RInteg{k_\perp } \left[ \; ik_\perp \tilde{E}_{z,m} \left( \frac{m}{k_\perp r} J_m(k_\perp r) -
    J_m'(k_\perp r) \right) -2 k_z\tilde{E}_{+,m}J_{m+1}(k_\perp r) \;
\right] \\
\RInteg{k_\perp } \; 2\, \partial_t \tilde{B}_{-,m}  J_{m-1}(k_\perp r) \; =
\RInteg{k_\perp } \left[ \;
   ik_\perp \tilde{E}_{z,m} \left( \frac{m}{k_\perp r} J_m(k_\perp r) +
    J_m'(k_\perp r) \right)  + 2k_z\tilde{E}_{-,m}J_{m-1}(k_\perp r) \;
\right] \\
\RInteg{k_\perp }\; \partial_t \tilde{B}_{z,m}  J_{m}(k_\perp r) =
\RInteg{k_\perp } \left[ \; -ik_\perp \tilde{E}_{+,m}\left(\frac{m+1}{k_\perp r}J_{m+1}(k_\perp r) +
    J_{m+1}'(k_\perp r) \right) \right.\\
\qquad \left. - ik_\perp \tilde{E}_{-,m}\left(\frac{m-1}{k_\perp r}J_{m-1}(k_\perp r) -
    J_{m-1}'(k_\perp r) \right) \right] 
\end{align*}
We can now use the relations $\frac{m}{k_\perp r} J_m(k_\perp r) +
    J_m'(k_\perp r) = J_{m-1}(k_\perp r)$ and $\frac{m}{k_\perp r} J_m(k_\perp r) -
    J_m'(k_\perp r) = J_{m+1}(k_\perp r)$ (see relation 9.1.27 in
    \cite{Abramowitz}), and obtain :
\begin{align*}
\RInteg{k_\perp } \; 2 \,\partial_t \tilde{B}_{+,m}  J_{m+1}(k_\perp r) =
\RInteg{k_\perp } \left[ \; ik_\perp \tilde{E}_{z,m}\,
    J_{m+1}(k_\perp r) -2 k_z\tilde{E}_{+,m}J_{m+1}(k_\perp r) \;
\right] \\
\RInteg{k_\perp } \; 2\, \partial_t \tilde{B}_{-,m}  J_{m-1}(k_\perp r) \; =
\RInteg{k_\perp } \left[ \;
   ik_\perp \tilde{E}_{z,m} \,
    J_{m-1}(k_\perp r) + 2k_z\tilde{E}_{-,m}J_{m-1}(k_\perp r) \;
\right] \\
\RInteg{k_\perp }\; \partial_t \tilde{B}_{z,m}  J_{m}(k_\perp r) =
\RInteg{k_\perp } \left[ \; -ik_\perp \tilde{E}_{+,m} J_{m}(k_\perp r) - ik_\perp \tilde{E}_{-,m}\,J_{m}(k_\perp r) \right] 
\end{align*}
Each equation of the above system contains Bessel functions of only one
given order (either $m+1$, $m-1$ or $m$). This allows to separate the
different $k_\perp $ components, since it the the functions $J_n(k_\perp r)$, for a
fixed $n$ and different values of $k_\perp $, form a basis of the set of real functions:
\begin{align*}
2 \,\partial_t \tilde{B}_{+,m} =
ik_\perp \tilde{E}_{z,m} -2 k_z\tilde{E}_{+,m} \\
2\, \partial_t \tilde{B}_{-,m} = ik_\perp \tilde{E}_{z,m} \,
    + 2k_z\tilde{E}_{-,m} \\
 \partial_t \tilde{B}_{z,m} = -ik_\perp \tilde{E}_{+,m}  - ik_\perp \tilde{E}_{-,m}
\end{align*}

\section{PSATD scheme, in the Fourier-Bessel
  representation}
\label{sec:PSTADderiv}

We use a scheme very similar to that of \cite{Haber}. In this scheme the currents are considered constant over one timestep, and the charge density is considered linear in time.

\subsection{Expressions for $\tilde{B}_m$}

By combining \cref{eq:CircMaxwellp,eq:CircMaxwellm,eq:CircMaxwellz}
and \cref{eq:SpectCons}, one can find the propagation equations for $B$.
\begin{align*}
\partial_t^2 \tilde{B}_{+,m} + c^2(k_\perp ^2+k_z^2) \tilde{B}_{+,m} = 
\mu_0 c^2 \left( - \frac{ik_\perp }{2} \tilde{j}_{z,m} + k_z \tilde{j}_{+,m}
\right) \\
\partial_t^2 \tilde{B}_{-,m} + c^2(k_\perp ^2+k_z^2) \tilde{B}_{-,m} = 
\mu_0 c^2 \left( - \frac{ik_\perp }{2} \tilde{j}_{z,m} - k_z \tilde{j}_{+,m}
\right) \\
\partial_t^2 \tilde{B}_{z,m} + c^2(k_\perp ^2+k_z^2) \tilde{B}_{z,m} =
\mu_0c^2  (ik_\perp  \tilde{j}_{+,m} + ik_\perp \tilde{j}_{-,m} ) 
\end{align*}
Let us integrate these equations for $t\in [n\Delta t, (n+1)\Delta
t]$. In this interval, $\tilde{j}_m(t)$ is constant
and equal to $\tj_m{n+1/2}$, and thus the right-hand side of the above
equations is constant. Using Green functions, the
general solution of a differential equation of the form 
$\partial_t^2 f + \omega^2 f = g_0$, where $g_0$ is a constant, is 
\[ f(t) = f(t_0) \cos[\,\omega (t-t_0)\,] + \partial_t f (t_0) \frac{
  \sin[\,\omega (t-t_0)\,]  }{\omega} + \frac{g_0}{\omega^2} (1-
\cos[\,\omega (t-t_0)\,] ) \]  
We thus use the above expression, with $\omega^2 =c^2(k_\perp^2 +
k_z^2)$, to integrate the fields from $t_0 = n\Delta t$ to $t=(n+1)\Delta t$. In
particular, we use again the Maxwell equations
\cref{eq:CircMaxwellp,eq:CircMaxwellm,eq:CircMaxwellz} to obtain the
expression of $\partial_t \tilde{B}_{m} (t_0)$. This yields:
\begin{align*}
\tB{+}{n+1} = \; & C \tB{+}{n} - 
\frac{S}{\omega}\left(-\frac{ik_\perp }{2} \tE{z}{n} + k_z\tE{+}{n}
\right) + \mu_0 c^2\frac{1-C}{\omega^2} \left( -\frac{ik_\perp }{2}
  \tj{z}{n+1/2} + k_z \tj{+}{n+1/2} \right)& \\
\tB{-}{n+1} =\; & C \tB{-}{n} - 
\frac{S}{\omega}\left(- \frac{ik_\perp }{2} \tE{z}{n} - k_z\tE{-}{n}
\right) + \mu_0 c^2\frac{1-C}{\omega^2} \left( - \frac{ik_\perp }{2}
  \tj{z}{n+1/2} - k_z \tj{-}{n+1/2} \right) &\\
\tB{z}{n+1} =\; & C \tB{z}{n} - 
\frac{S}{\omega}\left(ik_\perp \tE{+}{n} + ik_\perp \tE{-}{n}
\right) + \mu_0 c^2\frac{1-C}{\omega^2} \left( ik_\perp
  \tj{+}{n+1/2} + ik_\perp \tj{-}{n+1/2} \right)&
\end{align*}
where $C = \cos(\omega \Delta t)$ and $S = \sin(\omega \Delta t) $.

\subsection{Expressions for $\tilde{E}_m$}

Similarly, when combining \cref{eq:CircMaxwellp,eq:CircMaxwellm,eq:CircMaxwellz}
and \cref{eq:SpectCons}, the propagation equations for $E$ are:
\begin{align*}
\partial_t^2 \tilde{E}_{+,m} + c^2(k_\perp^2 + k_z^2) \tilde{E}_{+,m}
= \frac{c^2}{\epsilon_0} \frac{k_\perp}{2} \tilde{\rho}_m -
\mu_0c^2 \partial_t\tilde{j}_{+,m} \\
\partial_t^2 \tilde{E}_{-,m} + c^2(k_\perp^2 + k_z^2) \tilde{E}_{-,m}
= - \frac{c^2}{\epsilon_0} \frac{k_\perp}{2} \tilde{\rho}_m -
\mu_0c^2 \partial_t\tilde{j}_{-,m} \\
\partial_t^2 \tilde{E}_{z,m} + c^2(k_\perp^2 + k_z^2) \tilde{E}_{z,m}
= - \frac{c^2}{\epsilon_0} i k_z \tilde{\rho}_m -
\mu_0c^2 \partial_t\tilde{j}_{z,m} 
\end{align*}
Let us again integrate these equations for $t\in [n\Delta t, (n+1)\Delta
t]$. In this interval, $\tilde{j}_m(t)$ is constant (thus its time
derivatives drop), and $\tilde{\rho}_m$ is linear in time. As a
consequence the right hand side is proportional to $\trho{n} +
(\trho{n+1}-\trho{n})(t-t_0)/\Delta t$. Using Green functions, the solution of 
$ \partial_t^2 f + \omega^2 f = \trho{n} + (\trho{n+1}-\trho{n})(t-t_0)/\Delta t $ is
\[ f(t) = f(t_0) \cos[\,\omega(t-t_0)\,] + \partial_tf (t_0)
\frac{\sin[\,\omega(t-t_0)\,]}{\omega} + \trho{n}\frac{1-
  \cos[\,\omega(t-t_0)\,]}{\omega^2} + \frac{\trho{n+1}-\trho{n}}{\omega^2}\left(
  \frac{t-t_0}{\Delta t} - \frac{\sin[\,\omega(t-t_0)\,]}{\omega
    \Delta t}
\right) \]
which, for $t=t_0 +\Delta t$, reduces to
\[ f(t_0 +\Delta t) = f(t_0) C + \partial_tf (t_0)
\frac{S}{\omega} + \trho{n}\frac{S-\omega\Delta t
  C}{\omega^3\Delta t} + \frac{\trho{n+1}}{\omega^2}\left(
  1 - \frac{S}{\omega\Delta t}\right) \]
Using the above expression, we obtain
\begin{align*}
\tE{+}{n+1} = \; & C \tE{+}{n} + 
\frac{S}{\omega}\left(-\frac{ik_\perp }{2} \tB{z}{n} + k_z\tB{+}{n}
- \mu_0 \tj{+}{n+1/2} \right) + \frac{c^2}{\epsilon_0}
\frac{k_\perp}{2}\left[ \trho{n}\frac{S-\omega\Delta t
  C}{\omega^3\Delta t} + \frac{\trho{n+1}}{\omega^2}\left(
  1 - \frac{S}{\omega\Delta t}\right) \right]  & \\
\tE{-}{n+1} =\; & C \tE{-}{n} +
\frac{S}{\omega}\left(- \frac{ik_\perp }{2} \tB{z}{n} - k_z\tB{-}{n}
- \mu_0 \tj{-}{n+1/2} \right) - \frac{c^2}{\epsilon_0}
\frac{k_\perp}{2}\left[ \trho{n}\frac{S-\omega\Delta t
  C}{\omega^3\Delta t} + \frac{\trho{n+1}}{\omega^2}\left(
  1 - \frac{S}{\omega\Delta t}\right) \right]  &\\
\tE{z}{n+1} =\; & C \tE{z}{n} + 
\frac{S}{\omega}\left(ik_\perp \tB{+}{n} + ik_\perp \tB{-}{n}
- \mu_0 \tj{z}{n+1/2} \right) - \frac{c^2}{\epsilon_0}
ik_z\left[ \trho{n}\frac{S-\omega\Delta t
  C}{\omega^3\Delta t} + \frac{\trho{n+1}}{\omega^2}\left(
  1 - \frac{S}{\omega\Delta t}\right) \right]  &
\end{align*}

\section{Discrete Hankel Transform}
\label{sec:HTMatrix}

Say we extend the formulas \cite{Guizar} to the case of a regular
grid in space, and to the case where the order of the transform and
that of the $k$ grid are not the same.

Reminder:
\[ \mathrm{DHT_n}[f] \,(k^m_{\perp,j}) = \sum_{p=0}^{N_r-1} M_{j,p}(n,m)
\,f(r_p) \qquad \mathrm{IDHT_n}[g] \, (r_j) = \sum_{p=0}^{N_r-1}
M^{-1}_{j,p}(n,m) \,g(k^m_{\perp,p}) \]
These matrices can be determined by a set of $N_r^2$ conditions. These
conditions can be found by imposing the value of the DHT for a set of  $N_r$ functions.

In our case, we impose that the DHT be equal to the exact HT for the eigenmodes of a cavity with
perfectly conducting boundary at $r_{max}$ ($E_z(r_{m},z) =
0$), since these physical eigenmodes should also be eigenmodes of our
PIC cycle. These eigenmodes have the following form:
\begin{align*}
E_z \;& \; \propto  J_m(k^m_{\perp,\ell} \,r)\,e^{ik_z z -im\theta} \Theta(r_{max}-r) \\
E_r -i E_\theta \;& \; \propto  J_{m+1}(k^m_{\perp,\ell} \,r) \,e^{ik_z z -im\theta} \Theta(r_{max}-r)\\
E_r +i E_\theta \;& \; \propto  J_{m-1}(k^m_{\perp,\ell} \,r) \,e^{ik_z z -im\theta} \Theta(r_{max}-r) 
\end{align*}
where $\Theta$ is the Heaviside function and where $k^m_{\perp,\ell} =
\alpha^m_\ell / r_{max}$, with $\alpha^m_j$ the $j$th strictly positive zero of
the Bessel function of order $m$. They form an othogonal basis of
functions for this chosen boundary condition. The exact Hankel transform of these modes, evaluated on the discrete set $\{
k^m_{\perp,j} \}$ reads (see relation 11.4.5 in \cite{Abramowitz}).
\begin{align*} 
\mathrm{HT}_{m}[ \; J_m(k^m_{\perp,\ell} \,r) \;] \,(k^m_{\perp,j} )
&\quad \equiv 2\pi \rInteg \; J_m (k^m_{\perp,\ell} r) J_m (k^m_{\perp,j} r)
&\quad = \pi\, r_{max}^2\,[ J_{m+1}(\alpha_j^m)]^2 \; \delta_{j,\ell} \\
\mathrm{HT}_{m+1}[ \; J_{m+1}(k^m_{\perp,\ell} \,r) \;] \,(k^m_{\perp,j} )
&\quad \equiv 2\pi \rInteg \; J_{m+1} (k^m_{\perp,\ell} r)
  J_{m+1}(k^m_{\perp,j} r) 
&\quad = \pi\, r_{max}^2\,[ J_{m+1}(\alpha_j^m)]^2 \; \delta_{j,\ell} \\
\mathrm{HT}_{m-1}[ \; J_{m-1}(k^m_{\perp,\ell} \,r) \;] \,(k^m_{\perp,j} )
& \quad \equiv 2\pi \rInteg \; J_{m-1} (k^m_{\perp,\ell} r) J_{m-1} (k^m_{\perp,j}
  r) 
&\quad = \pi\, r_{max}^2\,[ J_{m+1}(\alpha_j^m)]^2 \; \delta_{j,\ell}
\end{align*}
\textbf{Explain more why we want these functions to be
  preserved. Explain why we take (m+1) and (m-1)}

Here we impose that the DHT of these functions has the same value for $(j,\ell) \in \{ 0,
..., N_r - 1\}$
\begin{align*}
\mathrm{DHT}_{m}[ \; J_m(k^m_{\perp,\ell} \,r) \;] \,(k^m_{\perp,j}) 
&\quad \equiv \sum_{p=0}^{N_r-1} M_{j,p}(m,m)
  J_m(k^m_{\perp,\ell}\,r_p) 
&\quad = \pi\, r_{max}^2\,[ J_{m+1}(\alpha_j^m)]^2 \; \delta_{j,\ell} \\
\mathrm{DHT}_{m+1}[ \; J_{m+1}(k^m_{\perp,\ell} \,r) \;]\,(k^m_{\perp,j})
&\quad \equiv \sum_{p=0}^{N_r-1} M_{j,p}(m+1,m)
  J_{m+1}(k^m_{\perp,\ell}\,r_p) 
&\quad = \pi\, r_{max}^2\,[ J_{m+1}(\alpha_j^m)]^2 \; \delta_{j,\ell} \\
\mathrm{DHT}_{m-1}[ \; J_{m-1}(k^m_{\perp,\ell} \,r) \;] \,(k^m_{\perp,j})
& \quad \equiv \sum_{p=0}^{N_r-1} M_{j,p}(m-1,m)
 J_{m-1}(k^m_{\perp,\ell}\,r_p) 
&\quad = \pi\, r_{max}^2\,[ J_{m+1}(\alpha_j^m)]^2 \; \delta_{j,\ell}
\end{align*}
The above relations impose $N_r^2$ conditions on the matrices
$M(m,m)$, $M(m+1,m)$, $M(m-1,m)$ and thus allows one to completely
determine them. In fact, a closer look at the above relations shows
that the inverse matrices $M^{-1}(m,m)$, $M^{-1}(m+1,m)$,
$M^{-1}(m-1,m)$ can be directly extracted from the above relations,
since these inverse matrices are defined by the relation $\sum_p M_{j,p}(n,m)
M^{-1}_{p,\ell}(n,m) = \delta_{j,\ell}$. Indeed, from the above
relations, one can directly infer:
\[ M^{-1}_{p,\ell}(m,m) = \frac{ J_m(k^m_{\perp,\ell}\,r_p) } { \pi\,
  r_{max}^2\,[ J_{m+1}(\alpha_j^m)]^2  } \]
\[ M^{-1}_{p,\ell}(m \pm 1,m) = \frac{ J_{m\pm 1}(k^m_{\perp,\ell}\,r_p) } { \pi\,
  r_{max}^2\,[ J_{m+1}(\alpha_j^m)]^2  } \]
The matrices $M(m,m)$, $M(m+1,m)$, $M(m-1,m)$ can then be extracted,
by numerically inverting the matrices  $M^{-1}(m,m)$, $M^{-1}(m+1,m)$,
$M^{-1}(m-1,m)$ given by the above expressions.

\bibliography{Bibliography}
\bibliographystyle{plain}

\end{document}  
