
\section{Derivation of the Fourier-Bessel representation}
\label{sec:CircTrans}

In order to derive the representation \cref{eq:CircBwTrans} we have to
distinguish the fields that are well-defined everywhere in space (like
$E_x$, $E_y$, $E_z$, $B_x$, $B_y$, $B_z$) and thus have a regular
Fourier representation, from those that are ill-defined at $r=0$ (like $E_r$, $E_\theta$, $B_r$, $B_\theta$).

\subsection{Fields that are well-defined everywhere}

Let $F_u$ be a field that is well-defined everywhere in space
(typically $F$ is $E$, $B$ or $J$ and $u$ is $x$, $y$ or $z$). Its Fourier representation
is thus given by \cref{eq:CartBwTrans,eq:CartFwTrans}:
\begin{align*}
F_u(\vec{r}) = \frac{1}{(2\pi)^{3/2}}\Integ{k_x} \,\Integ{k_y}\,
\Integ{k_z} \; \hat{F_u}(\vec{k}) e^{i(k_x x + k_y y + k_z z)} \\
\hat{F_u}(\vec{k})  = \frac{1}{(2\pi)^{3/2}}\Integ{x} \,\Integ{y}\,
\Integ{z} \; F_u(\vec{r}) e^{-i(k_x x + k_y y + k_z z)} 
\end{align*}
Using the change of variable $k_x=k\cos(\phi)$, $k_y = k\sin(\phi)$,
$x=r\cos(\theta)$, $y=r\sin(\theta)$, this becomes
 \begin{align*}
F_u(\vec{r}) = \frac{1}{(2\pi)^{3/2}}\Integ{k_z} \,\RInteg{k}\,
\TInteg{\phi} \; \hat{F_u}(\vec{k})
e^{i(k r \cos(\theta-\phi) + k_z z)} \\
\hat{F_u}(\vec{k})   = \frac{1}{(2\pi)^{3/2}}\Integ{z} \,\RInteg{r}\,
\Integ{\theta} \; F_u(\vec{r}) e^{-i(k r \cos(\theta-\phi) + k_z z)} 
\end{align*}
We now use the relation $e^{ikr\cos(\theta-\phi)} =
\sum_{m=-\infty}^{\infty} i^m J_m(kr) e^{im(\phi-\theta)}$ in these equations, to obtain
\begin{align*}
F_u (\vec{r})  = \sum_{m=-\infty}^{\infty} \frac{1}{(2\pi)^{3/2}}\Integ{k_z} \,\RInteg{k}
\TInteg{\phi} \; i^m \hat{F_u}(\vec{k}) \;
J_m(kr) e^{-im(\theta-\phi) + ik_z z} \\
\hat{F_u}(\vec{k})   =  \sum_{m=-\infty}^{\infty} \frac{1}{(2\pi)^{3/2}}\Integ{z} \,\RInteg{r}
\TInteg{\theta} \;\; (-i)^m F_u(\vec{r}) \; J_m(kr) e^{-im(\phi-\theta) -ik_z z} 
\end{align*}
We now define $\tilde{F}_{u,m}(k_z,k) = \frac{1}{(2\pi)^{3/2}}\int_0^{2\pi}
\mathrm{d}\phi \; i^m \hat{F_u}(\vec{k})
e^{im\phi}$. This results in the following equations :
\begin{align*}
F_u(\vec{r}) = \sum_{m=-\infty}^{\infty} \Integ{k_z}
\RInteg{k}\; \tilde{F}_{u,m}(k_z,k) \; J_m(kr) e^{-im\theta + ik_z z} 
\\
\tilde{F}_{u,m}(k_z,k) = \frac{1}{(2\pi)^2} \Integ{z} \RInteg{r}
\TInteg{\theta} \;F_u(\vec{r})\; J_m(kr) e^{-im\theta
 - i k_z z}
\end{align*}
These equations correspond to \cref{eq:CircBwTransu,eq:CircFwTransu}.

\subsection{Fields that are ill-defined at $r=0$}

Let us now consider fields of the type $E_r$, $B_r$ or $J_r$, which we
denote generally by $F_r$. We have :
\[ F_r = \cos(\theta) F_x + \sin(\theta) F_y 
= \frac{F_x - iF_y}{2}e^{i\theta} + \frac{F_x +
  iF_y}{2}e^{-i\theta} \] 
Using \cref{eq:CircBwTransu,eq:CircFwTransu}, this leads to
\begin{align} 
F_r =  \sum_{m=-\infty}^{\infty} \Integ{k_z}\,\RInteg{k}\;
\left(  J_m(kr) \frac{\tilde{F}_{x,m} -
    i\tilde{F}_{y,m}}{2}e^{-i(m-1)\theta +ik_z z} + J_m(kr)
  \frac{\tilde{F}_{x,m} +   i\tilde{F}_{y,m}}{2}e^{-i(m+1)\theta +
    ik_z z} \right) \\
= \sum_{m=-\infty}^{\infty} \Integ{k_z}\,\RInteg{k}\;
\left(  J_{m+1}(kr) \frac{\tilde{F}_{x,m+1} -
    i\tilde{F}_{y,m+1}}{2}e^{-im\theta +ik_z z} + J_{m-1}(kr)
  \frac{\tilde{F}_{x,m-1} +   i\tilde{F}_{y,m-1}}{2}e^{-im\theta +
    ik_z z} \right) 
\end{align}
where we relabeled the dummy variable $m$ in the above sums. Let us
thus define $\tilde{F}_{-,m} = \frac{\tilde{F}_{x,m-1} +
    i\tilde{F}_{y,m-1}}{2}$ and $\tilde{F}_{+,m} = \frac{\tilde{F}_{x,m+1} -
    i\tilde{F}_{y,m+1}}{2}$. This results in:
\begin{equation} 
F_r(\vec{r}) = \sum_{m=-\infty}^{\infty} \Integ{k_z}\,\RInteg{k}\;
\left( \tilde{F}_{+,m}\; J_{m+1}(kr) +\tilde{F}_{-,m}\; J_{m-1}(kr)
\right)  e^{-im\theta +ik_z z}
\end{equation}

With the same definitions and the same method, it is also easy to show that:
\begin{equation} 
F_\theta(\vec{r}) = \sum_{m=-\infty}^{\infty} \Integ{k_z}\,\RInteg{k}\;
i\left( \tilde{F}_{+,m}\; J_{m+1}(kr) - \tilde{F}_{-,m}\; J_{m-1}(kr)
\right)  e^{-im\theta +ik_z z}
\end{equation}

\section{Maxwell equations for the spectral coefficients}
\label{sec:SpectMaxwell}

In this section, let us derive the Maxwell equations for the spectral
coefficients \cref{eq:CircMaxwellp,eq:CircMaxwellm,eq:CircMaxwellz}
from the Maxwell equations written in cylindrical coordinates \cref{eq:CircMaxwellr,eq:CircMaxwellt,eq:CircMaxwellzz}.

When replacing the Fourier-Bessel decomposition
(\cref{eq:CircBwTransu,eq:CircBwTransr,eq:CircBwTranst}) in the
Maxwell equations \cref{eq:CircMaxwellr,eq:CircMaxwellt,eq:CircMaxwellzz}, we
first notice that the modes proportional to $e^{-im\theta +ik_z z}$ for different
values of $m$ and $k_z$ are not coupled. These different modes can
thus be treated separately. The same cannot be said of the modes
corresponding to different values of $k$, since they may be coupled
through the Bessel functions $J_m(kr)$ and their derivatives. 
In the following, we write only the equations corresponding to $\partial_t \vec{B} =
-\vec{\nabla}\times \vec{E}$, since the equation $\partial_t \vec{E} =
\vec{\nabla}\times\vec{B} - \mu_0 \vec{j}$ can be treated very
similarly. These equations become
\begin{align*}
\RInteg{k} \left[ \; \partial_t \tilde{B}_{+,m}  J_{m+1}(kr)
  + \partial_t \tilde{B}_{-,m}  J_{m-1}(kr) \; \right] =
\RInteg{k} \left[ \; \tilde{E}_{z,m} \frac{im}{r} J_m(kr) -
  k_z\tilde{E}_{+,m}J_{m+1}(kr) + k_z\tilde{E}_{-,m}J_{m-1}(kr) \;
\right] \\
\RInteg{k} \left[ \; \partial_t \tilde{B}_{+,m}  J_{m+1}(kr)
  - \partial_t \tilde{B}_{-,m}  J_{m-1}(kr) \; \right] =
\RInteg{k} \left[ \; -k_z\tilde{E}_{+,m}J_{m+1}(kr) -
  k_z\tilde{E}_{-,m}J_{m-1}(kr) - ik\tilde{E}_{z,m} J_m'(kr) \;
\right] \\
\RInteg{k}\; \partial_t \tilde{B}_{z,m}  J_{m}(kr) =
\RInteg{k} \left[ \; -ik\tilde{E}_{+,m}\left(\frac{J_{m+1}(kr)}{kr} +
    J_{m+1}'(kr) \right) + ik\tilde{E}_{-,m}\left(\frac{J_{m-1}(kr)}{kr} +
    J_{m-1}'(kr) \right) \right. \\
\qquad \left.- \frac{im}{r} \left( E_{+,m} J_{m+1}(kr) +
    E_{-,m} J_{m-1}(kr) \right) \;\right]
\end{align*}
By taking the sum and difference of the first two equations, and by
rearranging the third equation, we obtain:
\begin{align*}
\RInteg{k} \; 2 \,\partial_t \tilde{B}_{+,m}  J_{m+1}(kr) =
\RInteg{k} \left[ \; ik\tilde{E}_{z,m} \left( \frac{m}{kr} J_m(kr) -
    J_m'(kr) \right) -2 k_z\tilde{E}_{+,m}J_{m+1}(kr) \;
\right] \\
\RInteg{k} \; 2\, \partial_t \tilde{B}_{-,m}  J_{m-1}(kr) \; =
\RInteg{k} \left[ \;
   ik\tilde{E}_{z,m} \left( \frac{m}{kr} J_m(kr) +
    J_m'(kr) \right)  + 2k_z\tilde{E}_{-,m}J_{m-1}(kr) \;
\right] \\
\RInteg{k}\; \partial_t \tilde{B}_{z,m}  J_{m}(kr) =
\RInteg{k} \left[ \; -ik\tilde{E}_{+,m}\left(\frac{m+1}{kr}J_{m+1}(kr) +
    J_{m+1}'(kr) \right) \right.\\
\qquad \left. - ik\tilde{E}_{-,m}\left(\frac{m-1}{kr}J_{m-1}(kr) -
    J_{m-1}'(kr) \right) \right] 
\end{align*}
We can now use the relations $\frac{m}{kr} J_m(kr) +
    J_m'(kr) = J_{m-1}(kr)$ and $\frac{m}{kr} J_m(kr) -
    J_m'(kr) = J_{m+1}(kr)$ (see relation 9.1.27 in
    \cite{Abramowitz}), and obtain :
\begin{align*}
\RInteg{k} \; 2 \,\partial_t \tilde{B}_{+,m}  J_{m+1}(kr) =
\RInteg{k} \left[ \; ik\tilde{E}_{z,m}\,
    J_{m+1}(kr) -2 k_z\tilde{E}_{+,m}J_{m+1}(kr) \;
\right] \\
\RInteg{k} \; 2\, \partial_t \tilde{B}_{-,m}  J_{m-1}(kr) \; =
\RInteg{k} \left[ \;
   ik\tilde{E}_{z,m} \,
    J_{m-1}(kr) + 2k_z\tilde{E}_{-,m}J_{m-1}(kr) \;
\right] \\
\RInteg{k}\; \partial_t \tilde{B}_{z,m}  J_{m}(kr) =
\RInteg{k} \left[ \; -ik\tilde{E}_{+,m} J_{m}(kr) - ik\tilde{E}_{-,m}\,J_{m}(kr) \right] 
\end{align*}
Each equation of the above system contains Bessel functions of only one
given order (either $m+1$, $m-1$ or $m$). This allows to separate the
different $k$ components, since it the the functions $J_n(kr)$, for a
fixed $n$ different values of $k$, form a basis of the set of real functions:
\begin{align*}
2 \,\partial_t \tilde{B}_{+,m} =
ik\tilde{E}_{z,m} -2 k_z\tilde{E}_{+,m} \\
2\, \partial_t \tilde{B}_{-,m} = ik\tilde{E}_{z,m} \,
    + 2k_z\tilde{E}_{-,m} \\
 \partial_t \tilde{B}_{z,m} = -ik\tilde{E}_{+,m}  - ik\tilde{E}_{-,m}
\end{align*}

