
\section{Derivation of the Fourier-Bessel representation}
\label{sec:CircTrans}

In order to derive the representation \cref{eq:CircBwTrans} we have to
distinguish the fields that are well-defined everywhere in space (like
$E_x$, $E_y$, $E_z$, $B_x$, $B_y$, $B_z$) and thus have a regular
Fourier representation, from those that are ill-defined at $r=0$ (like $E_r$, $E_\theta$, $B_r$, $B_\theta$).

\subsection{Fields that are well-defined everywhere}

Let $F_u$ be a field that is well-defined everywhere in space
(typically $F$ is $E$, $B$ or $J$ and $u$ is $x$, $y$ or $z$). Its Fourier representation
is thus given by \cref{eq:CartBwTrans,eq:CartFwTrans}:
\begin{align*}
F_u(\vec{r}) = \frac{1}{(2\pi)^{3/2}}\Integ{k_x} \,\Integ{k_y}\,
\Integ{k_z} \; \hat{F_u}(\vec{k}) e^{i(k_x x + k_y y + k_z z)} \\
\hat{F_u}(\vec{k})  = \frac{1}{(2\pi)^{3/2}}\Integ{x} \,\Integ{y}\,
\Integ{z} \; F_u(\vec{r}) e^{-i(k_x x + k_y y + k_z z)} 
\end{align*}
Using the change of variable $k_x=k_\perp\cos(\phi)$, $k_y = k_\perp\sin(\phi)$,
$x=r\cos(\theta)$, $y=r\sin(\theta)$, this becomes
 \begin{align*}
F_u(\vec{r}) = \frac{1}{(2\pi)^{3/2}}\Integ{k_z} \,\RInteg{k_\perp}\,
\TInteg{\phi} \; \hat{F_u}(\vec{k})
e^{i(k_\perp r \cos(\theta-\phi) + k_z z)} \\
\hat{F_u}(\vec{k})   = \frac{1}{(2\pi)^{3/2}}\Integ{z} \,\RInteg{r}\,
\Integ{\theta} \; F_u(\vec{r}) e^{-i(k_\perp r \cos(\theta-\phi) + k_z z)} 
\end{align*}
We now use the relation $e^{ik_\perp r\cos(\theta-\phi)} =
\sum_{m=-\infty}^{\infty} i^m J_m(k_\perp r) e^{im(\phi-\theta)}$ in these equations, to obtain
\begin{align*}
F_u (\vec{r})  = \sum_{m=-\infty}^{\infty} \frac{1}{(2\pi)^{3/2}}\Integ{k_z} \,\RInteg{k_\perp }
\TInteg{\phi} \; i^m \hat{F_u}(\vec{k}) \;
J_m(k_\perp r) e^{-im(\theta-\phi) + ik_z z} \\
\hat{F_u}(\vec{k})   =  \sum_{m=-\infty}^{\infty} \frac{1}{(2\pi)^{3/2}}\Integ{z} \,\RInteg{r}
\TInteg{\theta} \;\; (-i)^m F_u(\vec{r}) \; J_m(k_\perp r) e^{-im(\phi-\theta) -ik_z z} 
\end{align*}
We now define $\tilde{F}_{u,m}(k_z,k_\perp ) = \frac{1}{(2\pi)^{3/2}}\int_0^{2\pi}
\mathrm{d}\phi \; i^m \hat{F_u}(\vec{k})
e^{im\phi}$. This results in the following equations :
\begin{align*}
F_u(\vec{r}) = \sum_{m=-\infty}^{\infty} \Integ{k_z}
\RInteg{k_\perp }\; \tilde{F}_{u,m}(k_z,k_\perp ) \; J_m(k_\perp r) e^{-im\theta + ik_z z} 
\\
\tilde{F}_{u,m}(k_z,k_\perp ) = \frac{1}{(2\pi)^2} \Integ{z} \RInteg{r}
\TInteg{\theta} \;F_u(\vec{r})\; J_m(k_\perp r) e^{-im\theta
 - i k_z z}
\end{align*}
These equations correspond to \cref{eq:CircBwTransu,eq:CircFwTransu}.

\subsection{Fields that are ill-defined at $r=0$}

Let us now consider fields of the type $E_r$, $B_r$ or $J_r$, which we
denote generally by $F_r$. We have :
\[ F_r = \cos(\theta) F_x + \sin(\theta) F_y 
= \frac{F_x - iF_y}{2}e^{i\theta} + \frac{F_x +
  iF_y}{2}e^{-i\theta} \] 
Using \cref{eq:CircBwTransu,eq:CircFwTransu}, this leads to
\begin{align} 
F_r =  \sum_{m=-\infty}^{\infty} \Integ{k_z}\,\RInteg{k_\perp }\;
\left(  J_m(k_\perp r) \frac{\tilde{F}_{x,m} -
    i\tilde{F}_{y,m}}{2}e^{-i(m-1)\theta +ik_z z} + J_m(k_\perp r)
  \frac{\tilde{F}_{x,m} +   i\tilde{F}_{y,m}}{2}e^{-i(m+1)\theta +
    ik_z z} \right) \\
= \sum_{m=-\infty}^{\infty} \Integ{k_z}\,\RInteg{k_\perp }\;
\left(  J_{m+1}(k_\perp r) \frac{\tilde{F}_{x,m+1} -
    i\tilde{F}_{y,m+1}}{2}e^{-im\theta +ik_z z} + J_{m-1}(k_\perp r)
  \frac{\tilde{F}_{x,m-1} +   i\tilde{F}_{y,m-1}}{2}e^{-im\theta +
    ik_z z} \right) 
\end{align}
where we relabeled the dummy variable $m$ in the above sums. Let us
thus define $\tilde{F}_{-,m} = \frac{\tilde{F}_{x,m-1} +
    i\tilde{F}_{y,m-1}}{2}$ and $\tilde{F}_{+,m} = \frac{\tilde{F}_{x,m+1} -
    i\tilde{F}_{y,m+1}}{2}$. This results in:
\begin{equation} 
F_r(\vec{r}) = \sum_{m=-\infty}^{\infty} \Integ{k_z}\,\RInteg{k_\perp }\;
\left( \tilde{F}_{+,m}\; J_{m+1}(k_\perp r) +\tilde{F}_{-,m}\; J_{m-1}(k_\perp r)
\right)  e^{-im\theta +ik_z z}
\end{equation}

With the same definitions and the same method, it is also easy to show that:
\begin{equation} 
F_\theta(\vec{r}) = \sum_{m=-\infty}^{\infty} \Integ{k_z}\,\RInteg{k_\perp }\;
i\left( \tilde{F}_{+,m}\; J_{m+1}(k_\perp r) - \tilde{F}_{-,m}\; J_{m-1}(k_\perp r)
\right)  e^{-im\theta +ik_z z}
\end{equation}

\section{Maxwell equations for the spectral coefficients}
\label{sec:SpectMaxwell}

In this section, let us derive the Maxwell equations for the spectral
coefficients \cref{eq:CircMaxwellp,eq:CircMaxwellm,eq:CircMaxwellz}
from the Maxwell equations written in cylindrical coordinates \cref{eq:CircMaxwellr,eq:CircMaxwellt,eq:CircMaxwellzz}.

When replacing the Fourier-Bessel decomposition
(\cref{eq:CircBwTransu,eq:CircBwTransr,eq:CircBwTranst}) in the
Maxwell equations \cref{eq:CircMaxwellr,eq:CircMaxwellt,eq:CircMaxwellzz}, we
first notice that the modes proportional to $e^{-im\theta +ik_z z}$ for different
values of $m$ and $k_z$ are not coupled. These different modes can
thus be treated separately. The same cannot be said of the modes
corresponding to different values of $k_\perp $, since they may be coupled
through the Bessel functions $J_m(k_\perp r)$ and their derivatives. 
In the following, we write only the equations corresponding to $\partial_t \vec{B} =
-\vec{\nabla}\times \vec{E}$, since the equation $c^{-2}\partial_t \vec{E} =
\vec{\nabla}\times\vec{B} - \mu_0 \vec{j}$ can be treated very
similarly. These equations become
\begin{align*}
\RInteg{k_\perp } \left[ \; \partial_t \tilde{B}_{+,m}  J_{m+1}(k_\perp r)
  + \partial_t \tilde{B}_{-,m}  J_{m-1}(k_\perp r) \; \right] =&& \\ 
\qquad \RInteg{k_\perp } \left[ \; \tilde{E}_{z,m} \frac{im}{r}
  J_m(k_\perp r) \right.-&\left.
  k_z\tilde{E}_{+,m}J_{m+1}(k_\perp r) + k_z\tilde{E}_{-,m}J_{m-1}(k_\perp r) \;
\right] & \\
\RInteg{k_\perp } \left[ \; \partial_t \tilde{B}_{+,m}  J_{m+1}(k_\perp r)
  - \partial_t \tilde{B}_{-,m}  J_{m-1}(k_\perp r) \; \right] =&& \\
 \RInteg{k_\perp } \left[ \; -k_z\tilde{E}_{+,m}J_{m+1}(k_\perp r)
 \right.-&\left.  k_z\tilde{E}_{-,m}J_{m-1}(k_\perp r) - ik_\perp \tilde{E}_{z,m} J_m'(k_\perp r) \;\right] \\
\RInteg{k_\perp }\; \partial_t \tilde{B}_{z,m}  J_{m}(k_\perp r) =
\RInteg{k_\perp } \left[ \; -ik_\perp
  \tilde{E}_{+,m}\right.&\left(\frac{J_{m+1}(k_\perp r)}{k_\perp r} +
    J_{m+1}'(k_\perp r) \right) + &\\
\left. ik_\perp \tilde{E}_{-,m}\left(\frac{J_{m-1}(k_\perp r)}{k_\perp r} +
    J_{m-1}'(k_\perp r) \right) \right.-&\left. \frac{im}{r} \left( E_{+,m} J_{m+1}(k_\perp r) +
    E_{-,m} J_{m-1}(k_\perp r) \right) \;\right] 
\end{align*}
By taking the sum and difference of the first two equations, and by
rearranging the third equation, we obtain:
\begin{align*}
\RInteg{k_\perp } \; 2 \,\partial_t \tilde{B}_{+,m}  J_{m+1}(k_\perp r) =
\RInteg{k_\perp } \left[ \; ik_\perp \tilde{E}_{z,m} \left( \frac{m}{k_\perp r} J_m(k_\perp r) -
    J_m'(k_\perp r) \right) -2 k_z\tilde{E}_{+,m}J_{m+1}(k_\perp r) \;
\right] \\
\RInteg{k_\perp } \; 2\, \partial_t \tilde{B}_{-,m}  J_{m-1}(k_\perp r) \; =
\RInteg{k_\perp } \left[ \;
   ik_\perp \tilde{E}_{z,m} \left( \frac{m}{k_\perp r} J_m(k_\perp r) +
    J_m'(k_\perp r) \right)  + 2k_z\tilde{E}_{-,m}J_{m-1}(k_\perp r) \;
\right] \\
\RInteg{k_\perp }\; \partial_t \tilde{B}_{z,m}  J_{m}(k_\perp r) =
\RInteg{k_\perp } \left[ \; -ik_\perp \tilde{E}_{+,m}\left(\frac{m+1}{k_\perp r}J_{m+1}(k_\perp r) +
    J_{m+1}'(k_\perp r) \right) \right.\\
\qquad \left. - ik_\perp \tilde{E}_{-,m}\left(\frac{m-1}{k_\perp r}J_{m-1}(k_\perp r) -
    J_{m-1}'(k_\perp r) \right) \right] 
\end{align*}
We can now use the relations $\frac{m}{k_\perp r} J_m(k_\perp r) +
    J_m'(k_\perp r) = J_{m-1}(k_\perp r)$ and $\frac{m}{k_\perp r} J_m(k_\perp r) -
    J_m'(k_\perp r) = J_{m+1}(k_\perp r)$ (see relation 9.1.27 in
    \cite{Abramowitz}), and obtain :
\begin{align*}
\RInteg{k_\perp } \; 2 \,\partial_t \tilde{B}_{+,m}  J_{m+1}(k_\perp r) =
\RInteg{k_\perp } \left[ \; ik_\perp \tilde{E}_{z,m}\,
    J_{m+1}(k_\perp r) -2 k_z\tilde{E}_{+,m}J_{m+1}(k_\perp r) \;
\right] \\
\RInteg{k_\perp } \; 2\, \partial_t \tilde{B}_{-,m}  J_{m-1}(k_\perp r) \; =
\RInteg{k_\perp } \left[ \;
   ik_\perp \tilde{E}_{z,m} \,
    J_{m-1}(k_\perp r) + 2k_z\tilde{E}_{-,m}J_{m-1}(k_\perp r) \;
\right] \\
\RInteg{k_\perp }\; \partial_t \tilde{B}_{z,m}  J_{m}(k_\perp r) =
\RInteg{k_\perp } \left[ \; -ik_\perp \tilde{E}_{+,m} J_{m}(k_\perp r) - ik_\perp \tilde{E}_{-,m}\,J_{m}(k_\perp r) \right] 
\end{align*}
Each equation of the above system contains Bessel functions of only one
given order (either $m+1$, $m-1$ or $m$). This allows to separate the
different $k_\perp $ components, since it the the functions $J_n(k_\perp r)$, for a
fixed $n$ and different values of $k_\perp $, form a basis of the set of real functions:
\begin{align*}
2 \,\partial_t \tilde{B}_{+,m} =
ik_\perp \tilde{E}_{z,m} -2 k_z\tilde{E}_{+,m} \\
2\, \partial_t \tilde{B}_{-,m} = ik_\perp \tilde{E}_{z,m} \,
    + 2k_z\tilde{E}_{-,m} \\
 \partial_t \tilde{B}_{z,m} = -ik_\perp \tilde{E}_{+,m}  - ik_\perp \tilde{E}_{-,m}
\end{align*}

\section{PSATD scheme, in the Fourier-Bessel
  representation}

Intro on the analytical approach. Say very similar to Haber. Ensures
no numerical dispersion in vacuum.
Say that the fields rho and j are linear and constant respectively

\subsection{Expressions for $\tilde{B}_m$}

By combining \cref{eq:CircMaxwellp,eq:CircMaxwellm,eq:CircMaxwellz}
and \cref{eq:SpectCons}, one can find the propagation equations for $B$.
\begin{align*}
\partial_t^2 \tilde{B}_{+,m} + c^2(k_\perp ^2+k_z^2) \tilde{B}_{+,m} = 
\mu_0 c^2 \left( - \frac{ik_\perp }{2} \tilde{j}_{z,m} + k_z \tilde{j}_{+,m}
\right) \\
\partial_t^2 \tilde{B}_{-,m} + c^2(k_\perp ^2+k_z^2) \tilde{B}_{-,m} = 
\mu_0 c^2 \left( - \frac{ik_\perp }{2} \tilde{j}_{z,m} - k_z \tilde{j}_{+,m}
\right) \\
\partial_t^2 \tilde{B}_{z,m} + c^2(k_\perp ^2+k_z^2) \tilde{B}_{z,m} =
\mu_0c^2  (ik_\perp  \tilde{j}_{+,m} + ik_\perp \tilde{j}_{-,m} ) 
\end{align*}
Let us integrate these equations for $t\in [n\Delta t, (n+1)\Delta
t]$. In this interval, $\tilde{j}_m(t)$ is constant
and equal to $\tj_m{n+1/2}$, and thus the right-hand side of the above
equations is constant. Using Green functions, the
general solution of a differential equation of the form 
$\partial_t^2 f + \omega^2 f = g_0$, where $g_0$ is a constant, is 
\[ f(t) = f(t_0) \cos[\,\omega (t-t_0)\,] + \partial_t f (t_0) \frac{
  \sin[\,\omega (t-t_0)\,]  }{\omega} + \frac{g_0}{\omega^2} (1-
\cos[\,\omega (t-t_0)\,] ) \]  
We thus use the above expression, with $\omega^2 =c^2(k_\perp^2 +
k_z^2)$, to integrate the fields from $t_0 = n\Delta t$ to $t=(n+1)\Delta t$. In
particular, we use again the Maxwell equations
\cref{eq:CircMaxwellp,eq:CircMaxwellm,eq:CircMaxwellz} to obtain the
expression of $\partial_t \tilde{B}_{m} (t_0)$. This yields:
\begin{align*}
\tB{+}{n+1} = \; & C \tB{+}{n} - 
\frac{S}{\omega}\left(-\frac{ik_\perp }{2} \tE{z}{n} + k_z\tE{+}{n}
\right) + \mu_0 c^2\frac{1-C}{\omega^2} \left( -\frac{ik_\perp }{2}
  \tj{z}{n+1/2} + k_z \tj{+}{n+1/2} \right)& \\
\tB{-}{n+1} =\; & C \tB{-}{n} - 
\frac{S}{\omega}\left(- \frac{ik_\perp }{2} \tE{z}{n} - k_z\tE{-}{n}
\right) + \mu_0 c^2\frac{1-C}{\omega^2} \left( - \frac{ik_\perp }{2}
  \tj{z}{n+1/2} - k_z \tj{-}{n+1/2} \right) &\\
\tB{z}{n+1} =\; & C \tB{z}{n} - 
\frac{S}{\omega}\left(ik_\perp \tE{+}{n} + ik_\perp \tE{-}{n}
\right) + \mu_0 c^2\frac{1-C}{\omega^2} \left( ik_\perp
  \tj{+}{n+1/2} + ik_\perp \tj{-}{n+1/2} \right)&
\end{align*}
where $C = \cos(\omega \Delta t)$ and $S = \sin(\omega \Delta t) $.

\subsection{Expressions for $\tilde{E}_m$}

Similarly, when combining \cref{eq:CircMaxwellp,eq:CircMaxwellm,eq:CircMaxwellz}
and \cref{eq:SpectCons}, the propagation equations for $E$ are:
\begin{align*}
\partial_t^2 \tilde{E}_{+,m} + c^2(k_\perp^2 + k_z^2) \tilde{E}_{+,m}
= \frac{c^2}{\epsilon_0} \frac{k_\perp}{2} \tilde{\rho}_m -
\mu_0c^2 \partial_t\tilde{j}_{+,m} \\
\partial_t^2 \tilde{E}_{-,m} + c^2(k_\perp^2 + k_z^2) \tilde{E}_{-,m}
= - \frac{c^2}{\epsilon_0} \frac{k_\perp}{2} \tilde{\rho}_m -
\mu_0c^2 \partial_t\tilde{j}_{-,m} \\
\partial_t^2 \tilde{E}_{z,m} + c^2(k_\perp^2 + k_z^2) \tilde{E}_{z,m}
= - \frac{c^2}{\epsilon_0} i k_z \tilde{\rho}_m -
\mu_0c^2 \partial_t\tilde{j}_{z,m} 
\end{align*}
Let us again integrate these equations for $t\in [n\Delta t, (n+1)\Delta
t]$. In this interval, $\tilde{j}_m(t)$ is constant (thus its time
derivatives drop), and $\tilde{\rho}_m$ is linear in time. As a
consequence the right hand side is proportional to $\trho{n} +
(\trho{n+1}-\trho{n})(t-t_0)/\Delta t$. Using Green functions, the solution of 
$ \partial_t^2 f + \omega^2 f = \trho{n} + (\trho{n+1}-\trho{n})(t-t_0)/\Delta t $ is
\[ f(t) = f(t_0) \cos[\,\omega(t-t_0)\,] + \partial_tf (t_0)
\frac{\sin[\,\omega(t-t_0)\,]}{\omega} + \trho{n}\frac{1-
  \cos[\,\omega(t-t_0)\,]}{\omega^2} + \frac{\trho{n+1}-\trho{n}}{\omega^2}\left(
  \frac{t-t_0}{\Delta t} - \frac{\sin[\,\omega(t-t_0)\,]}{\omega
    \Delta t}
\right) \]
which, for $t=t_0 +\Delta t$, reduces to
\[ f(t_0 +\Delta t) = f(t_0) C + \partial_tf (t_0)
\frac{S}{\omega} + \trho{n}\frac{S-\omega\Delta t
  C}{\omega^3\Delta t} + \frac{\trho{n+1}}{\omega^2}\left(
  1 - \frac{S}{\omega\Delta t}\right) \]
Using the above expression, we obtain
\begin{align*}
\tE{+}{n+1} = \; & C \tE{+}{n} + 
\frac{S}{\omega}\left(-\frac{ik_\perp }{2} \tB{z}{n} + k_z\tB{+}{n}
- \mu_0 \tj{+}{n+1/2} \right) + \frac{c^2}{\epsilon_0}
\frac{k_\perp}{2}\left[ \trho{n}\frac{S-\omega\Delta t
  C}{\omega^3\Delta t} + \frac{\trho{n+1}}{\omega^2}\left(
  1 - \frac{S}{\omega\Delta t}\right) \right]  & \\
\tE{-}{n+1} =\; & C \tE{-}{n} +
\frac{S}{\omega}\left(- \frac{ik_\perp }{2} \tB{z}{n} - k_z\tB{-}{n}
- \mu_0 \tj{-}{n+1/2} \right) - \frac{c^2}{\epsilon_0}
\frac{k_\perp}{2}\left[ \trho{n}\frac{S-\omega\Delta t
  C}{\omega^3\Delta t} + \frac{\trho{n+1}}{\omega^2}\left(
  1 - \frac{S}{\omega\Delta t}\right) \right]  &\\
\tE{z}{n+1} =\; & C \tE{z}{n} + 
\frac{S}{\omega}\left(ik_\perp \tB{+}{n} + ik_\perp \tB{-}{n}
- \mu_0 \tj{z}{n+1/2} \right) - \frac{c^2}{\epsilon_0}
ik_z\left[ \trho{n}\frac{S-\omega\Delta t
  C}{\omega^3\Delta t} + \frac{\trho{n+1}}{\omega^2}\left(
  1 - \frac{S}{\omega\Delta t}\right) \right]  &
\end{align*}
