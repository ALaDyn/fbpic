
\section{Derivation of the Fourier-Bessel representation}
\label{sec:CircTrans}

In order to derive the representation \cref{eq:CircBwTrans} we have to
distinguish the fields that are well-defined everywhere in space (like
$E_x$, $E_y$, $E_z$, $B_x$, $B_y$, $B_z$) and thus have a regular
Fourier representation, from those that are ill-defined at $r=0$ (like $E_r$, $E_\theta$, $B_r$, $B_\theta$).

\subsection{Fields that are well-defined everywhere}

Let $F_u$ be a field that is well-defined everywhere in space
(typically $F$ is $E$, $B$ or $J$ and $u$ is $x$, $y$ or $z$). Its Fourier representation
is thus given by \cref{eq:CartBwTrans,eq:CartFwTrans}:
\begin{align*}
F_u(x,y,z) = \frac{1}{(2\pi)^{3/2}}\Integ{k_x} \,\Integ{k_y}\,
\Integ{k_z} \; \hat{F_u}(k_x,k_y,k_z) e^{i(k_x x + k_y y + k_z z)} \\
\hat{F_u}(k_x,k_y,k_z)  = \frac{1}{(2\pi)^{3/2}}\Integ{x} \,\Integ{y}\,
\Integ{z} \; F_u(x,y,z) e^{-i(k_x x + k_y y + k_z z)} 
\end{align*}
Using the change of variable $k_x=k\cos(\phi)$, $k_y = k\sin(\phi)$,
$x=r\cos(\theta)$, $y=r\sin(\theta)$, this becomes
 \begin{align*}
F_u = \frac{1}{(2\pi)^{3/2}}\Integ{k_z} \,\RInteg{k}\,
\TInteg{\phi} \; \hat{F_u}(k\cos(\phi),k\sin(\phi),k_z)
e^{i(k r \cos(\theta-\phi) + k_z z)} \\
\hat{F_u}  = \frac{1}{(2\pi)^{3/2}}\Integ{z} \,\RInteg{r}\,
\Integ{\theta} \; F_u(r\cos(\theta),r\sin(\theta),z) e^{-i(k r \cos(\theta-\phi) + k_z z)} 
\end{align*}
We now use the relation $e^{ikr\cos(\theta-\phi)} =
\sum_{m=-\infty}^{\infty} i^m J_m(kr) e^{im(\phi-\theta)}$ in these equations, to obtain
\begin{align*}
F_u = \sum_{m=-\infty}^{\infty} \frac{1}{(2\pi)^{3/2}}\Integ{k_z} \,\RInteg{k}
\TInteg{\phi} \; i^m \hat{F_u}(k\cos(\phi),k\sin(\phi),k_z) \;
J_m(kr) e^{-im(\theta-\phi) + ik_z z} \\
\hat{F_u}  =  \sum_{m=-\infty}^{\infty} \frac{1}{(2\pi)^{3/2}}\Integ{z} \,\RInteg{r}
\TInteg{\theta} \;\; (-i)^m F_u(r\cos(\theta),r\sin(\theta),z) \; J_m(kr) e^{-im(\phi-\theta) -ik_z z} 
\end{align*}
We now define $\tilde{F}_{u,m}(k_z,k) = \frac{1}{(2\pi)^{3/2}}\int_0^{2\pi}
\mathrm{d}\phi \; i^m \hat{F_u}(k\cos(\phi),k\sin(\phi),k_z)
e^{im\phi}$. This results in the following equations :
\begin{align*}
F_u = \sum_{m=-\infty}^{\infty} \Integ{k_z}
\RInteg{k}\; \tilde{F}_{u,m}(k_z,k) \; J_m(kr) e^{-im\theta + ik_z z} 
\\
\tilde{F}_{u,m} = \frac{1}{(2\pi)^2} \Integ{z} \RInteg{r}
\TInteg{\theta} \;F_u(r\cos(\theta), r\sin(\theta),z)\; J_m(kr) e^{-im\theta
 - i k_z z}
\end{align*}
These equations correspond to \cref{eq:CircBwTransz,eq:CircFwTransz}.

\subsection{Fields that are ill-defined at $r=0$}

Let us now consider fields of the type $E_r$, $B_r$ or $J_r$, which we
denote generally by $F_r$. We have :
\[ F_r = \cos(\theta) F_x + \sin(\theta) F_y 
= \frac{F_x - iF_y}{2}e^{i\theta} + \frac{F_x +
  iF_y}{2}e^{-i\theta} \] 
Using \cref{eq:CircBwTransu,eq:CircFwTransu}, this leads to
\begin{align} 
F_r =  \sum_{m=-\infty}^{\infty} \Integ{k_z}\,\RInteg{k}\;
\left(  J_m(kr) \frac{\tilde{F}_{x,m} -
    i\tilde{F}_{y,m}}{2}e^{-i(m-1)\theta +ik_z z} + J_m(kr)
  \frac{\tilde{F}_{x,m} +   i\tilde{F}_{y,m}}{2}e^{-i(m+1)\theta +
    ik_z z} \right) \\
= \sum_{m=-\infty}^{\infty} \Integ{k_z}\,\RInteg{k}\;
\left(  J_{m+1}(kr) \frac{\tilde{F}_{x,m+1} -
    i\tilde{F}_{y,m+1}}{2}e^{-im\theta +ik_z z} + J_{m-1}(kr)
  \frac{\tilde{F}_{x,m-1} +   i\tilde{F}_{y,m-1}}{2}e^{-im\theta +
    ik_z z} \right) 
\end{align}
where we relabeled the dummy variable $m$ in the above sums. Let us
thus define $\tilde{F}_{+,m} = \frac{\tilde{F}_{x,m+1} -
    i\tilde{F}_{y,m+1}}{2}$ and $\tilde{F}_{-,m} = \frac{\tilde{F}_{x,m-1} +
    i\tilde{F}_{y,m-1}}{2}$. This results in:
\begin{align} 
F_r = \sum_{m=-\infty}^{\infty} \Integ{k_z}\,\RInteg{k}\;
\left(  J_{m+1}(kr)\tilde{F}_{+,m} + J_{m-1}(kr)\tilde{F}_{-,m}
  \right) e^{-im\theta +ik_z z}
\end{align}

\section{Derivation of the Maxwell equations in the Fourier-Bessel representation}
